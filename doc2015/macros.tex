% ----- listings

\definecolor{ckeyword}{HTML}{7F0055}
\definecolor{ccomment}{HTML}{3F7F5F}
\definecolor{cstring}{HTML}{2A0099}
% name 7F0055

\lstdefinelanguage{Scala}%
{morekeywords={abstract,%
  case,catch,char,class,%
  def,else,extends,final,finally,for,forSome,%
  if,import,implicit,%
  match,module,%
  new,null,%
  object,override,%
  package,private,protected,public,%
  for,public,return,super,sealed,%
  this,throw,trait,try,type,%
  val,var,%
  with,while,%
  yield%
  },%
  sensitive,%
  morecomment=[l]//,%
  morecomment=[s]{/*}{*/},%
  morestring=[b]",%
  morestring=[b]',%
  showstringspaces=false%
}[keywords,comments,strings]%

\lstset{language=Scala,%
  mathescape=true,%
%  columns=[c]fixed,%
%  basewidth={0.5em, 0.40em},%
%  aboveskip=1pt,%\smallskipamount,
%  belowskip=1pt,%\negsmallskipamount,
  lineskip=-0.5pt,
  basewidth={0.54em, 0.4em},%
%  backgroundcolor=\color{listingbg},
  basicstyle=\footnotesize\ttfamily,
  keywordstyle={\color{ckeyword}\ttfamily\bfseries},
  commentstyle={\color{ccomment}\itshape},
  stringstyle={\color{cstring}},
  xleftmargin=1ex
}

% class EclipseStyle(Style):
%     """
%     Style similar to the style used in Eclipse IDEs. (Jevon)
%     """

%     default_style = ''

%     styles = {
%         Whitespace:             '#bbbbbb',

%         Comment:                   "nobold #3F7F5F",    # 63,127,95
%       Comment.Multiline:             "#3F5FBF",     # assume all multiline comments = Javadoc comments, 63,95,191
%     Comment.Preproc:           'italic #666',   # e.g. XML preprocessing instruction
  
%         # Comment.Preproc:           "noitalic",

%         Keyword:                   "bold #7F0055",    # 127,0,85
%         Keyword.Pseudo:            "#f00",
%         Keyword.Type:              "bold #7F0055",    # e.g. int

%         Operator:                   "#000",       # e.g. +, -, (, )
%         # Operator.Word:             "#00f",

%     Number:                 "#000",

% #     Name:                      "nobold #000",
%         Name.Class:                "nobold #000",
%         Name.Namespace:            "nobold #000",
%         Name.Exception:            "nobold #000",
%         Name.Entity:               "nobold #000",
%         Name.Tag:                  "bold #7F0055",    # XML/HTML tags - 127,0,85
%       Name.Function:             "nobold #000",
%         Name.Attribute:            "nobold #000",     # XML/HTML attributes
%         Name.Decorator:            "nobold #646464",      # @Annotations - 100,100,100
    
%         String:                    "#2A00FF",   # 42,0,255
%         #String.Interpol:           "bold",
%         #String.Escape:             "bold",

%         Generic.Heading:           "bold",
%         Generic.Subheading:        "bold",
%         Generic.Emph:              "italic",
%         Generic.Strong:            "bold",
%         Generic.Prompt:            "bold",
        
%         Error:                     "border:#FF0000"
%     }



\definecolor{listingbg}{RGB}{240, 240, 240}

\newcommand{\commentstyle}[1]{\slseries{#1}}
\newcommand{\keywordstyle}[1]{\bfseries{#1}}

\lstnewenvironment{listing}{\lstset{language=Scala}}{}
\lstnewenvironment{listingtiny}{\lstset{language=Scala,basicstyle=\scriptsize\ttfamily}}{}

\lstnewenvironment{minted}{\lstset{language=Scala}}{}


\newcommand{\code}[1]{\lstinline[language=Scala,columns=fixed,basicstyle=\ttfamily]|#1|}

\newcommand{\mint}[1]{\lstinline[language=Scala,columns=fixed,basicstyle=\ttfamily]|#1|}



% ----- packed items, so we don't waste space
\newenvironment{sitemize}{\vspace{-4pt}
\begin{itemize}
  \setlength{\itemsep}{1pt}
  \setlength{\parskip}{0pt}
  \setlength{\parsep}{0pt}
}{\end{itemize}}

\newenvironment{senumerate}{
\begin{enumerate}
  \setlength{\itemsep}{1pt}
  \setlength{\parskip}{0pt}
  \setlength{\parsep}{0pt}
}{\end{enumerate}}

\newcommand{\mypar}[1]{{\bf #1.}}

% ----- comments and todo

%\newcommand{\note}[1]{{\color{red}[#1]}}
%\newcommand{\todo}[1]{\note{TODO: #1}}

%\newcommand{\comment}[1]{}